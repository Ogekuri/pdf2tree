\documentclass[a4paper,11pt]{report}

% Encoding e lingua
\usepackage[utf8]{inputenc}
\usepackage[T1]{fontenc}
\usepackage[italian]{babel}

% Matematica
\usepackage{amsmath,amssymb}

% Grafica e immagini
\usepackage{graphicx}
\usepackage{tikz}
\usetikzlibrary{shapes.geometric, arrows.meta, positioning}
\usepackage{booktabs}
\usepackage{float}
\usepackage{mwe}       % immagini di esempio: example-image-a, ecc.
\usepackage[a4paper,left=1cm,right=1cm,top=1.5cm,bottom=2cm]{geometry}

% Link cliccabili nell'indice e nel testo
\usepackage{hyperref}

% Testo fittizio
\usepackage{lipsum}
\usepackage{fancyhdr}
\setlength{\headheight}{13.6pt}

\begin{document}

\title{Documento Casuale di Esempio}
\author{Generato automaticamente}
\date{\today}

\maketitle

\tableofcontents

\pagestyle{fancy}
\fancyhf{}
\fancyhead[L]{Generato automaticamente}
\fancyhead[C]{Documento Casuale di Esempio}
\fancyhead[R]{\nouppercase{\leftmark}}
\renewcommand{\headrulewidth}{0.4pt}
\fancyfoot[C]{\thepage}
\fancypagestyle{plain}{%
  \fancyhf{}%
  \fancyhead[L]{Generato automaticamente}%
  \fancyhead[C]{Documento Casuale di Esempio}%
  \fancyhead[R]{\nouppercase{\leftmark}}%
  \renewcommand{\headrulewidth}{0.4pt}%
  \fancyfoot[C]{\thepage}%
}

\clearpage

%====================================================
\chapter{Introduzione Stravagante}
%====================================================

\section{Motivazioni improbabili}

Questo documento non ha alcuno scopo pratico preciso, se non quello di mostrare
un albero di capitoli e sotto-capitoli con contenuti misti:
tabelle, immagini raster, figure vettoriali e formule matematiche.

\lipsum[1-2]

\section{Formule apparentemente profonde}

Consideriamo una funzione $f \colon \mathbb{R} \to \mathbb{R}$ definita da
\[
  f(x) = e^{-x^2}\sin(3x).
\]
Possiamo calcolare (o fingere di calcolare) la seguente integrale:
\[
  I = \int_{-\infty}^{+\infty} e^{-x^2}\sin(3x)\,dx.
\]

Un altro esempio di formula un po' casuale:
\[
  \sum_{n=1}^{\infty} \frac{(-1)^n}{n^2} = -\frac{\pi^2}{12}.
\]

Infine, un sistema di equazioni lineari:
\[
\begin{cases}
  2x + 3y - z = 1, \\
  -x + 4y + 5z = 0, \\
  x - y + 2z = 3.
\end{cases}
\]

%====================================================
\chapter{Dati Senza Senso}
%====================================================

\section{Tabella di numeri casuali}

Nel seguito, presentiamo una tabella di numeri perfettamente inutili,
che non rappresentano niente di reale.

\begin{table}[h]
  \centering
  \begin{tabular}{lccc}
    \toprule
    Etichetta & $x$ & $y$ & $z$ \\
    \midrule
    A & 3.14 & -1.00 & 42 \\
    B & 0.00 & 1.73  & -7 \\
    C & -2.50 & 0.99 & 13 \\
    D & 6.28 & -3.14 & 0 \\
    \bottomrule
  \end{tabular}
  \caption{Tabella totalmente priva di significato pratico.}
  \label{tab:nonsenso}
\end{table}

Possiamo anche collegare questi valori ad una pseudo-funzione:
\[
  g(t) = x \sin(t) + y \cos(t) + z,
\]
dove $(x,y,z)$ sono scelti a caso dalla Tabella~\ref{tab:nonsenso}.

\section{Ancora un po' di testo casuale}

\lipsum[3-4]

%====================================================
\chapter{Immagini Misteriose}
%====================================================

\section{Immagine raster di esempio}

Qui inseriamo una immagine “raster” di esempio. Se hai un file
\texttt{mia\_immagine.png}, puoi sostituire \texttt{example-image-a}
con il nome del tuo file.

\begin{figure}[h]
  \centering
  \includegraphics[width=0.7\textwidth]{example-image-a}
  \caption{Immagine raster di esempio (sostituibile con una tua PNG/JPG).}
  \label{fig:raster1}
\end{figure}

Come se non bastasse, ecco un'altra immagine:

\begin{figure}[h]
  \centering
  \includegraphics[width=0.5\textwidth]{example-image-b}
  \caption{Seconda immagine raster di esempio.}
  \label{fig:raster2}
\end{figure}

\clearpage

\section{Mini-esperimento numerico con immagini}

Immaginiamo che le figure \ref{fig:raster1} e \ref{fig:raster2} rappresentino
due misurazioni sperimentali di una certa grandezza (anche se non è vero).
Possiamo “definire” una distanza fittizia $d$ tra le due immagini come:
\[
  d = \sqrt{(x_1 - x_2)^2 + (y_1 - y_2)^2},
\]
dove $(x_1,y_1)$ e $(x_2,y_2)$ sono coordinate immaginarie associate alle figure.

%====================================================
\chapter{Figure Vettoriali e Geometria}
%====================================================

\section{Diagramma vettoriale con TikZ}

Ora passiamo a una figura vettoriale, generata direttamente in PDF tramite TikZ.

\begin{figure}[h]
  \centering
  \begin{tikzpicture}[scale=1.0]
    % Assi cartesiani
    \draw[->] (-0.5,0) -- (4.5,0) node[right] {$x$};
    \draw[->] (0,-0.5) -- (0,3.5) node[above] {$y$};

    % Vettore v
    \draw[thick] (0,0) -- (3,2) node[midway, above] {$\vec v$};
    \fill (3,2) circle (2pt) node[right]{$(3,2)$};

    % Proiezioni
    \draw[thick,dashed] (3,0) node[below]{$3$} -- (3,2);
    \draw[thick,dashed] (0,2) node[left]{$2$} -- (3,2);

    % Angolo
    \node at (1.5,0.3) {$\theta$};
    \draw (0.7,0) arc (0:34:0.7);
  \end{tikzpicture}
  \caption{Esempio di figura vettoriale generata con TikZ.}
  \label{fig:tikz}
\end{figure}

Questa figura è “vettoriale” nel senso che viene disegnata tramite istruzioni
di TikZ e quindi è scalabile senza perdita di qualità.

\section{Una formula legata alla figura}

Se interpretiamo $\vec v = (3,2)$ come un vettore nel piano,
la sua norma euclidea è
\[
  \|\vec v\| = \sqrt{3^2 + 2^2} = \sqrt{13}.
\]
L'angolo $\theta$ che forma con l'asse $x$ è dato da
\[
  \theta = \arctan\left(\frac{2}{3}\right).
\]

\lipsum[5]

%====================================================
\chapter{Formule Matematiche Casuali}
%====================================================

\section{Equazioni differenziali}

\lipsum[8]

Un'equazione differenziale ordinaria del secondo ordine:
\[
    \frac{d^2y}{dx^2} + 2\frac{dy}{dx} + y = \cos(x)
\]

E una equazione alle derivate parziali (l'equazione del calore):
\[
    \frac{\partial u}{\partial t} = \alpha \nabla^2 u = \alpha \left( \frac{\partial^2 u}{\partial x^2} + \frac{\partial^2 u}{\partial y^2} + \frac{\partial^2 u}{\partial z^2} \right)
\]

\section{Algebra lineare}

\lipsum[9]

Una matrice 3x3 casuale:
\[
A = \begin{pmatrix}
    a & b & c \\
    d & e & f \\
    g & h & i
\end{pmatrix}
\]

Il suo determinante è dato da:
\[
    \det(A) = a(ei - fh) - b(di - fg) + c(dh - eg)
\]

%====================================================
\chapter{Flow Chart Casuale}
%====================================================

\begin{figure}[p]
  \centering
  \begin{tikzpicture}[node distance=12mm, auto,>=Stealth, every node/.style={font=\small}]
    \tikzstyle{startstop} = [rectangle, rounded corners, minimum width=3cm, minimum height=8mm, text centered, draw=black, fill=gray!10]
    \tikzstyle{process} = [rectangle, minimum width=3cm, minimum height=8mm, text centered, draw=black, fill=blue!10]
    \tikzstyle{decision} = [diamond, aspect=2, text centered, draw=black, fill=green!10]
    \node[startstop] (start) {Inizio};
    \node[process, below=of start] (step1) {Step 1};
    \node[decision, below=of step1] (decide) {Condizione?};
    \node[process, below left=of decide] (step2) {Step 2};
    \node[process, below right=of decide] (step3) {Step 3};
    \node[startstop, below=of decide, yshift=-18mm] (stop) {Fine};
    \draw[->] (start) -- (step1);
    \draw[->] (step1) -- (decide);
    \draw[->] (decide) -- node[left] {Sì} (step2);
    \draw[->] (decide) -- node[right] {No} (step3);
    \draw[->] (step2) |- (stop);
    \draw[->] (step3) |- (stop);
  \end{tikzpicture}
  \caption{Flow chart casuale generato con TikZ.}
  \label{fig:flowchart}
\end{figure}

%====================================================
\chapter{Conti Matriciali Sconclusionati}
%====================================================

\section{Prodotti tra matrici 2x2}

Scegliamo due matrici $2 \times 2$ con numeri interi raccolti a caso:
\[
  M =
  \begin{pmatrix}
    2 & -1 \\
    3 & 4
  \end{pmatrix},
  \qquad
  N =
  \begin{pmatrix}
    1 & 5 \\
    -2 & 0
  \end{pmatrix}.
\]

Il prodotto $P = M \cdot N$ risulta:
\[
  P =
  \begin{pmatrix}
    4 & 10 \\
    -5 & 15
  \end{pmatrix}.
\]

\section{Determinanti e tracce a caso}

I determinanti delle tre matrici sono
\begin{align*}
  \det(M) &= 2 \cdot 4 - (-1) \cdot 3 = 11, \\
  \det(N) &= 1 \cdot 0 - 5 \cdot (-2) = 10, \\
  \det(P) &= 4 \cdot 15 - 10 \cdot (-5) = 110 = \det(M)\,\det(N).
\end{align*}

La traccia di $P$ è $\operatorname{tr}(P) = 4 + 15 = 19$, un numero
che non rivela segreti cosmici ma conferma che i conti tornano.

\lipsum[5]

%====================================================
\chapter{Conclusioni Inutili}
%====================================================

\section{Riflessioni finali}

\lipsum[6-7]

In conclusione, questo documento dimostra solo che si possono mischiare
contenuti completamente casuali in un unico file LaTeX:
tabelle, immagini, formule e figure vettoriali,
senza alcuna coerenza scientifica ma con una struttura formale ordinata.

\end{document}
